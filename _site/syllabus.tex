% Options for packages loaded elsewhere
\PassOptionsToPackage{unicode}{hyperref}
\PassOptionsToPackage{hyphens}{url}
\documentclass[letterpaper]{article}
\usepackage[
  margin=1in,
  % includefoot,
  % footskip=30pt,
]{geometry}
\usepackage{xcolor}
\usepackage{amsmath,amssymb}
\setcounter{secnumdepth}{-\maxdimen} % remove section numbering
\usepackage{iftex}
\ifPDFTeX
  \usepackage[T1]{fontenc}
  \usepackage[utf8]{inputenc}
  \usepackage{textcomp} % provide euro and other symbols
\else % if luatex or xetex
  \usepackage{unicode-math} % this also loads fontspec
  \defaultfontfeatures{Scale=MatchLowercase}
  \defaultfontfeatures[\rmfamily]{Ligatures=TeX,Scale=1}
\fi
\usepackage{lmodern}
\ifPDFTeX\else
  % xetex/luatex font selection
\fi
% Use upquote if available, for straight quotes in verbatim environments
\IfFileExists{upquote.sty}{\usepackage{upquote}}{}
\IfFileExists{microtype.sty}{% use microtype if available
  \usepackage[]{microtype}
  \UseMicrotypeSet[protrusion]{basicmath} % disable protrusion for tt fonts
}{}
\makeatletter
\@ifundefined{KOMAClassName}{% if non-KOMA class
  \IfFileExists{parskip.sty}{%
    \usepackage{parskip}
  }{% else
    \setlength{\parindent}{0pt}
    \setlength{\parskip}{6pt plus 2pt minus 1pt}}
}{% if KOMA class
  \KOMAoptions{parskip=half}}
\makeatother
\usepackage{longtable,booktabs,array}
\usepackage{calc} % for calculating minipage widths
% Correct order of tables after \paragraph or \subparagraph
\usepackage{etoolbox}
\makeatletter
\patchcmd\longtable{\par}{\if@noskipsec\mbox{}\fi\par}{}{}
\makeatother
% Allow footnotes in longtable head/foot
\IfFileExists{footnotehyper.sty}{\usepackage{footnotehyper}}{\usepackage{footnote}}
\makesavenoteenv{longtable}
% definitions for citeproc citations
\NewDocumentCommand\citeproctext{}{}
\NewDocumentCommand\citeproc{mm}{%
  \begingroup\def\citeproctext{#2}\cite{#1}\endgroup}
\makeatletter
 % allow citations to break across lines
 \let\@cite@ofmt\@firstofone
 % avoid brackets around text for \cite:
 \def\@biblabel#1{}
 \def\@cite#1#2{{#1\if@tempswa , #2\fi}}
\makeatother
\newlength{\cslhangindent}
\setlength{\cslhangindent}{1.5em}
\newlength{\csllabelwidth}
\setlength{\csllabelwidth}{3em}
\newenvironment{CSLReferences}[2] % #1 hanging-indent, #2 entry-spacing
 {\begin{list}{}{%
  \setlength{\itemindent}{0pt}
  \setlength{\leftmargin}{0pt}
  \setlength{\parsep}{0pt}
  % turn on hanging indent if param 1 is 1
  \ifodd #1
   \setlength{\leftmargin}{\cslhangindent}
   \setlength{\itemindent}{-1\cslhangindent}
  \fi
  % set entry spacing
  \setlength{\itemsep}{#2\baselineskip}}}
 {\end{list}}
\usepackage{calc}
\newcommand{\CSLBlock}[1]{\hfill\break\parbox[t]{\linewidth}{\strut\ignorespaces#1\strut}}
\newcommand{\CSLLeftMargin}[1]{\parbox[t]{\csllabelwidth}{\strut#1\strut}}
\newcommand{\CSLRightInline}[1]{\parbox[t]{\linewidth - \csllabelwidth}{\strut#1\strut}}
\newcommand{\CSLIndent}[1]{\hspace{\cslhangindent}#1}
\ifLuaTeX
\usepackage[bidi=basic]{babel}
\else
\usepackage[bidi=default]{babel}
\fi
\babelprovide[main,import]{english}
% get rid of language-specific shorthands (see #6817):
\let\LanguageShortHands\languageshorthands
\def\languageshorthands#1{}
\ifLuaTeX
  \usepackage[english]{selnolig} % disable illegal ligatures
\fi
\setlength{\emergencystretch}{3em} % prevent overfull lines
\providecommand{\tightlist}{%
  \setlength{\itemsep}{0pt}\setlength{\parskip}{0pt}}
\usepackage{bookmark}
\IfFileExists{xurl.sty}{\usepackage{xurl}}{} % add URL line breaks if available
\urlstyle{same}
\usepackage{hyperref}
\hypersetup{
  pdftitle={Computational Social Science for Sustainability (EBS 181/281) -- Comp. Soc. Sci. for Sustainability (EBS 181/281)},
  pdflang={en},
  hidelinks,
  pdfcreator={LaTeX via pandoc}}

\title{Computational Social Science for Sustainability (EBS 181/281)}
\author{}
\date{}

\begin{document}
\maketitle

% \phantomsection\label{quarto-document-content}
% \phantomsection\label{title-block-header}
% \section{Computational Social Science for Sustainability (EBS
% 181/281)}\label{computational-social-science-for-sustainability-ebs-181281}

% \subsection{Winter 2025}\label{winter-2025}
\vspace{-0.5in}
\subsubsection{Schedule and location}\label{schedule-and-location}

Monday lectures and Wednesday lab sections 11:30 AM - 12:50 PM in
\href{https://campus-map.stanford.edu/?srch=320-109\#}{320-109},
\href{https://studentservices.stanford.edu/calendar/academic-dates/stanford-academic-calendar-2024-2025\#winter25}{Winter
quarter 2025}.

\subsubsection{Instructor}\label{instructor}

Dr.~Matthew A. Turner, PhD (\href{mailto:maturner@stanford.edu}{email})

Feel free to call me by my first name.

Please see my \href{https://mt.digital}{professional site} to learn more
about me.

\subsubsection{Office hours}\label{office-hours}

\begin{itemize}
\tightlist
\item
  Monday and Tuesday 1 - 3 PM (Y2E2 352)
\item
  Thursday 3 - 4 PM (Zoom; link to be shared on Canvas)
\item
  By appointment
\end{itemize}

\subsubsection{Getting help}\label{getting-help}

I recommend seeking help early if students feel stuck or lost. Please
attend \hyperref[office-hours]{office hours},
\href{mailto:maturner@stanford.edu}{email me} to ask questions or set up
a time to meet, or seek the help from other students.

\subsection{Course overview}\label{course-overview}

The development and diffusion of sustainable innovations, cooperation
for sustainable resource management, and political polarization that can
undermine these, share a common explanation: these phenomena all emerge
from repeated social interactions between individuals over time. These
interactions take the form of social learning, social influence, or
strategic economic cooperation.

Take this course to learn how to develop computational ``experiments''
of repeated social interaction that can be used to design more effective
sustainability interventions and analyze behavioral data. Students will
learn transferrable technical skills in programming and mathematics.
Students will deepen their interdisciplinary understanding of from the
social and behavioral sciences. With these skills and understanding,
students will be empowered to create research products that analyze and
evaluate potential sustainability interventions using computation.

\subsubsection{Learning goals}\label{learning-goals}

Through participation in lecture and lab sections and completion of
course activities, students will

\begin{itemize}
\tightlist
\item
  learn modern approaches to scientific modeling and statistical
  analysis of social behavior
\item
  learn to develop, implement, and analyze their own models for
  designing sustainability interventions
\item
  learn to write effective research papers using the Introduction,
  Model, Analysis, Discussion structure
\item
  improve their R programming skills, including the popular
  \href{https://www.tidyverse.org/}{\texttt{tidyverse}}, for simulation
  modeling and data science (see the R for Data Science
  book){[}https://r4ds.hadley.nz/{]} for more details.
\end{itemize}

\subsubsection{Expectations}\label{expectations}

Understanding mutual expectations can help everyone succeed. There are
things you can expect of me and things I expect of enrolled students.

\paragraph{What you can expect of me}\label{what-you-can-expect-of-me}

I will do my best to promote an encouraging, safe, and fair learning
environment to promote student success. I will strive to understand and
support student career goals coming from a diversity of life
experiences.

You can expect I will be eager to help when needed, especially if you
start from very little to no experience with math and programming. I
understand that these subjects can cause anxiety for some.

\paragraph{Expectations of all
students}\label{expectations-of-all-students}

Students are expected attend all scheduled course meetings unless there
are extenuating circumstances. Please email if this occurs. Students are
expected to seek help if they are struggling or stuck.

\paragraph{Expectations of graduate
students}\label{expectations-of-graduate-students}

Graduate students will be required to complete an extra exercise on each
problem set that will be extra credit for undergraduates. They will be
held to higher standards for clarity, structure, and technical detail in
midterm and final projects.

\subsubsection{Course materials}\label{course-materials}

Students will need a laptop or otherwise portable computer to bring to
the Wednesday lab sections. There are a number of readings from journals
and books (see the \href{./\#Calendar}{Calendar} below), but these are
either available through Stanford Libraries, or if not I will provide
PDF copies via Canvas.

\subsection{Course structure}\label{course-structure}

Each week will have a Monday lecture on topics in computational social
science for sustainability. Wednesday meetings will focus on developing
programming, analysis, and writing skills in an interactive lab-section
setting. In these Wednesday sections students will be introduced to
problem sets and midterm and final projects, and have time to ask
questions of and work together with the instructor and peers.

\subsubsection{Coursework and Grading}\label{coursework-and-grading}

Students will be evaluated based on their completion of six assignments
worth 100 points total: four problem sets (10 points each), a midterm
project (20 points), and a final project (40 points). Undergraduate
students will have the opportunity for bonus points on each problem set.

\begin{itemize}
\tightlist
\item
  \textbf{Problem sets:} There will be four problem sets introduced on
  Wednesdays during the computing lab section. Students will also have
  the opportunity to work together and ask the instructor questions
  during other lab sections before each assignment is due. (10 points
  per problem set)
\item
  \textbf{Midterm project:} Students will write a report on how they
  will use a model from the course to address a sustainability problem
  of interest. The midterm project will be used as a foundation for the
  final project. (20 points)
\item
  \textbf{Final project:} Students will expand on their midterm project,
  performing a detailed model analysis and discussing the implications
  of their results for designing sustainability interventions. (40
  points)
\end{itemize}

\subsubsection{Late coursework policy}\label{late-coursework-policy}

If there is a family or health emergency or other acute distress please
contact me to make arrangements to submit late work without penalty.
Otherwise the following policy applies:

\begin{itemize}
\tightlist
\item
  Problem sets up to 72 hours late can receive 50\% credit.
\item
  Problem sets up to one week late can receive 20\% credit.
\item
  The midterm project may be submitted up to one week late to receive
  50\% credit.
\item
  The final project may not be submitted late.
\item
  No credit will be given for work beyond one week late.
\end{itemize}

\subsection{Lecture and lab calendar and topics
outline}\label{lecture-and-lab-calendar-and-topics-outline}

\subsubsection{Calendar}\label{calendar}

In the calendar below, PS stands for \emph{problem set}. Subject to
change.

\begin{longtable}[]{@{}
  >{\raggedright\arraybackslash}p{(\linewidth - 6\tabcolsep) * \real{0.2500}}
  >{\raggedright\arraybackslash}p{(\linewidth - 6\tabcolsep) * \real{0.2500}}
  >{\raggedright\arraybackslash}p{(\linewidth - 6\tabcolsep) * \real{0.2500}}
  >{\raggedright\arraybackslash}p{(\linewidth - 6\tabcolsep) * \real{0.2500}}@{}}
\toprule\noalign{}
\begin{minipage}[b]{\linewidth}\raggedright
Week
\end{minipage} & \begin{minipage}[b]{\linewidth}\raggedright
Topic
\end{minipage} & \begin{minipage}[b]{\linewidth}\raggedright
Coursework
\end{minipage} & \begin{minipage}[b]{\linewidth}\raggedright
Readings
\end{minipage} \\
\midrule\noalign{}
\endhead
\bottomrule\noalign{}
\endlastfoot
1, M 1/6 and W 1/8 & What is social science, why \emph{computation}, and
how these can promote sustainability in socio-ecological systems. &
\begin{minipage}[t]{\linewidth}\raggedright
\begin{itemize}
\tightlist
\item
  \textbf{PS 1, diffusion via social learning:} How many individuals
  need to be know about beneficial sustainable innovations so that they
  spread throughout a population? \textbf{Due 1/22 at 11:30 AM}
\end{itemize}
\end{minipage} & \begin{minipage}[t]{\linewidth}\raggedright
\begin{itemize}
\tightlist
\item
  {A. Pisor, Lansing, and Magargal
  (\hyperref[ref-PisorLansingMagargal2023]{2023})}
\item
  {Ostrom (\hyperref[ref-Ostrom2014]{2014})}
\item
  {Cox, Arnold, and Tomás (\hyperref[ref-Cox2010]{2010})}
\end{itemize}
\end{minipage} \\
2, M 1/13 and W 1/15 & The effect of social networks on sustainable
innovation development and diffusion. & &
\begin{minipage}[t]{\linewidth}\raggedright
\begin{itemize}
\tightlist
\item
  {A. C. Pisor, Borgerhoff Mulder, and Smith
  (\hyperref[ref-Pisor2024]{2024})}
\item
  {Centola (\hyperref[ref-Centola2022]{2022})}
\item
  {Derex and Boyd (\hyperref[ref-Derex2016]{2016})}
\end{itemize}
\end{minipage} \\
3, M 1/20 and W 1/22 & How asymmetric preferences for within-group
interaction can create sustainability-promoting social networks, and how
to measure this in the real world. &
\begin{minipage}[t]{\linewidth}\raggedright
\begin{itemize}
\tightlist
\item
  \textbf{PS 1 due 1/22 at 11:30 AM}
\item
  \textbf{PS 2, how information and opinions spread in different social
  networks:} How careful timing of informational interventions can help
  limit polarization. \textbf{Due 2/5 at 11:30 AM}
\end{itemize}
\end{minipage} & \begin{minipage}[t]{\linewidth}\raggedright
\begin{itemize}
\tightlist
\item
 {Matthew A. Turner et al.
(\hyperref[ref-Turner2023]{2023})} 
\end{itemize}
\end{minipage} \\
4, W 1/29 (no class M 1/27) & Social influence represented as forces
causing opinion dynamics. & & \begin{minipage}[t]{\linewidth}\raggedright
\begin{itemize}
\tightlist
\item {Matthew A. Turner and Smaldino (\hyperref[ref-Turner2018]{2018})} 
\item Smaldino (2023; Ch. 6) \end{itemize}
\end{minipage} \\
5, M 2/3 and W 2/5 & Opinion dynamics in the context of sustainability.
Experimental design and measurement in opinion dynamics experiments.  &
\begin{minipage}[t]{\linewidth}\raggedright
\begin{itemize}
\tightlist
\item
  \textbf{PS 2 due W 2/5 at 11:30 AM}
\item
  \textbf{Midterm project announced W 2/5, due 2/24 at 11:59 PM}
\end{itemize}
\end{minipage} & \begin{minipage}[t]{\linewidth}\raggedright
\begin{itemize}
\tightlist
\item
  {Liddell and Kruschke (\hyperref[ref-Liddell2018]{2018})}
\item
  {Galesic et al. (\hyperref[ref-GalesicEtAl2023]{2023})}
\end{itemize}
\end{minipage} \\
6, M 2/10 and W 2/12 & Common-pool resource management dilemmas: when
and why do people cooperate? (Part I) &
\begin{minipage}[t]{\linewidth}\raggedright
\begin{itemize}
\tightlist
\item
  \textbf{PS 3, modeling and measuring socio-ecological dilemmas:}
  applications include groundwater management and
  reforestation.\textbf{Due W 2/26 at 11:30 AM}
\end{itemize}
\end{minipage} & \begin{minipage}[t]{\linewidth}\raggedright
\begin{itemize}
\tightlist
\item
  {Nowak (\hyperref[ref-Nowak2006]{2006})}
\item
  {J. Andrews and Borgerhoff Mulder (\hyperref[ref-Andrews2018]{2018})}
\end{itemize}
\end{minipage} \\
7, W 2/19 (no class M 2/17) & Common-pool resource management dilemmas:
when and why do people cooperate? (Part II) &
\begin{minipage}[t]{\linewidth}\raggedright
\begin{itemize}
\tightlist
\item
  \textbf{Final project announced, due M 3/17 at 11:59 PM}
\end{itemize}
\end{minipage} & \begin{minipage}[t]{\linewidth}\raggedright
\begin{itemize}
\tightlist
\item
  {Tavoni et al. (\hyperref[ref-Tavoni2011]{2011})}
\item
  {Jackson (\hyperref[ref-Jackson2008]{2008})}, Ch. 9, especially 9.2
  and 9.3
\end{itemize}
\end{minipage} \\
8, M 2/24 and W 2/26 & How to perform and report computational social
model analyses. & \begin{minipage}[t]{\linewidth}\raggedright
\begin{itemize}
\tightlist
\item
  \textbf{PS 3 due W 2/26 at 11:30 AM}
\item
  \textbf{PS 4, the evolution of institutions for sustainable property
  rights:} with example of promoting carbon-capture farming. \textbf{Due
  W 3/12 at 11:30 AM}
\end{itemize}
\end{minipage} & \\
9, M 3/3 and W 3/5 & Common-pool resource management dilemmas: when and
why do people cooperate? (Part II) & &
\begin{minipage}[t]{\linewidth}\raggedright
\begin{itemize}
\tightlist
\item
  {Waring et al. (\hyperref[ref-Waring2015]{2015})}
\item
  {Waring, Goff, and Smaldino (\hyperref[ref-Waring2017]{2017})}
\item
  {Jeffrey Andrews et al. (\hyperref[ref-Andrews2024]{2024})}
\end{itemize}
\end{minipage} \\
10, M 3/10 and W 3/12 & Review: A look back at how computational social
science can promote sustainability, through the lens of the Price
equation. & \begin{minipage}[t]{\linewidth}\raggedright
\begin{itemize}
\tightlist
\item
  \textbf{PS 4 Due W 3/12}
\item
  \textbf{Final project due W 3/5 at 11:30 AM}
\end{itemize}
\end{minipage} & \begin{minipage}[t]{\linewidth}\raggedright
\begin{itemize}
\tightlist
\item
  {Deffner et al. (\hyperref[ref-Deffner2024]{2024})}
\item
  {Bak-Coleman et al. (\hyperref[ref-Bak-Coleman2021]{2021})}
\end{itemize}
\end{minipage} \\
\end{longtable}

\subsubsection{Course outline}\label{course-outline}

\begin{enumerate}
\def\labelenumi{\arabic{enumi}.}
\item
  Computational social science can help design sustainability
  interventions. Social science theory provides models of repeated human
  interaction over time that can be used, for example, to represent
  \href{https://earthbound.report/2018/01/15/elinor-ostroms-8-rules-for-managing-the-commons/}{Ostrom's
  eight ``design principles'' for sustainable socio-ecological systems}.

  \begin{itemize}
  \tightlist
  \item
    Lab: Introducing the \emph{80\% success rate} exercise, ``How much
    advertising is necessary for an 80\% success rate in spreading a
    sustainable innovation in groups, given population size and average
    number of acquaintances of people in the group?''
  \end{itemize}
\item
  How human psychology, groups, and social networks can promote or
  inhibit the diffusion of sustainable innovations, Part I: single-group
  social networks.

  \begin{itemize}
  \tightlist
  \item
    Lab: Could innovation-supporting social networks also promote
    inequality {(\hyperref[ref-Moser2023]{Moser and Smaldino 2023})}?
  \end{itemize}
\item
  How human psychology, groups, and social networks can promote or
  inhibit the diffusion of sustainable innovations, Part II: two-group
  (or more) social networks, i.e., \emph{metapopulation} social
  networks.

  \begin{itemize}
  \item
    Lab I: 80\% success rate exercise with two-group social networks
    defined by each group's \emph{homophily} level, i.e., tendency of
    group members to interact with others from their own group
    {(\hyperref[ref-Turner2023]{Matthew A. Turner et al. 2023})}.
  \item
    Lab II: Use stochastic block model to infer networks from data
    {(\hyperref[ref-DeBacco2023]{De Bacco et al. 2023};
    \hyperref[ref-Ross2024]{Ross, McElreath, and Redhead 2024})}.
  \end{itemize}
\item
  Social influence: understanding the effect of rhetoric as a force that
  acts on opinions and beliefs. How to measure opinion dynamics and

  \begin{itemize}
  \tightlist
  \item
    Lab I: When is polarization path-dependent and therefore possible to
    avoid {(\hyperref[ref-Turner2018]{Matthew A. Turner and Smaldino
    2018})}?
  \item
    Lab II: Opinion dynamics measurement depends on accurate inference
    using categorical (Likert-style) observational data
    {(\hyperref[ref-Liddell2018]{Liddell and Kruschke 2018})}.
  \end{itemize}
\item
  The emergence of cooperation via reciprocity: application to
  groundwater sustainability. How to predict and restrict potential
  free-riding based on marginal utility in managing common pool
  resources using game theory (see
  {(\hyperref[ref-Jackson]{\textbf{Jackson?}})}).

  \begin{itemize}
  \tightlist
  \item
    Lab I: ``Groundwater sharing dilemma'' (as we'll call it, though
    it's just a re-telling of the famous
    \href{https://en.wikipedia.org/wiki/Prisoner\%27s_dilemma}{prisoners'
    dilemma})
  \item
    Lab II: Agent-based model of behavioral study of ``avoidance of
    disastrous climate change in a public goods game'' by {Tavoni et al.
    (\hyperref[ref-Tavoni2011]{2011})}
  \end{itemize}
\item
  Ideal institutions support human cooperation by balancing variation
  and maintenance of beneficial behaviors within and between stakeholder
  groups {(\hyperref[ref-Richerson2016d]{Richerson et al. 2016};
  \hyperref[ref-Waring2015]{Waring et al. 2015})}. Example: sustainable
  agricultural practices like crop switching
  {(\hyperref[ref-Waring2023]{Waring et al. 2023};
  \hyperref[ref-Kling2024]{Kling et al. 2024})}.

  \begin{itemize}
  \tightlist
  \item
    Lab I: The evolution of property rights supports sustainability
    {(\hyperref[ref-Waring2017]{Waring, Goff, and Smaldino 2017})}. What
    sorts of social networks evolve? Could alternatives better promote
    or inhibit the development and diffusion of innovations?
  \end{itemize}
\end{enumerate}

\phantomsection\label{quarto-appendix}
\phantomsection\label{quarto-bibliography}
\subsection{References}\label{references}

\phantomsection\label{refs}
\begin{CSLReferences}{1}{0}
\bibitem[\citeproctext]{ref-Andrews2018}
Andrews, J., and M. Borgerhoff Mulder. 2018. {``{Cultural group
selection and the design of REDD+: insights from Pemba}.''}
\emph{Sustainability Science} 13 (1): 93--107.
\url{https://doi.org/10.1007/s11625-017-0489-2}.

\bibitem[\citeproctext]{ref-Andrews2024}
Andrews, Jeffrey, Matthew Clark, Vicken Hillis, and Monique Borgerhoff
Mulder. 2024. {``{The cultural evolution of collective property rights
for sustainable resource governance}.''} \emph{Nature Sustainability}.
\url{https://doi.org/10.1038/s41893-024-01290-1}.

\bibitem[\citeproctext]{ref-Bak-Coleman2021}
Bak-Coleman, Joseph B., Mark Alfano, Wolfram Barfuss, Carl T. Bergstrom,
Miguel A. Centeno, Iain D. Couzin, Jonathan F. Donges, et al. 2021.
{``{Stewardship of global collective behavior}.''} \emph{Proceedings of
the National Academy of Sciences of the United States of America} 118
(27): 1--10. \url{https://doi.org/10.1073/pnas.2025764118}.

\bibitem[\citeproctext]{ref-Centola2022}
Centola, Damon. 2022. {``{The network science of collective
intelligence}.''} \emph{Trends in Cognitive Sciences} 26 (11): 923--41.
\url{https://doi.org/10.1016/j.tics.2022.08.009}.

\bibitem[\citeproctext]{ref-Cox2010}
Cox, Michael, Gwen Arnold, and Sergio Villamayor Tomás. 2010. {``{A
review of design principles for community-based natural resource
management}.''} \emph{Ecology and Society} 15 (4).
\url{https://doi.org/10.5751/ES-03704-150438}.

\bibitem[\citeproctext]{ref-DeBacco2023}
De Bacco, Caterina, Martina Contisciani, Jonathan Cardoso-Silva, Hadiseh
Safdari, Gabriela Lima Borges, Diego Baptista, Tracy Sweet, et al. 2023.
{``{Latent network models to account for noisy, multiply reported social
network data}.''} \emph{Journal of the Royal Statistical Society Series
A: Statistics in Society} 186 (3): 355--75.
\url{https://doi.org/10.1093/jrsssa/qnac004}.

\bibitem[\citeproctext]{ref-Deffner2024}
Deffner, Dominik, Natalia Fedorova, Jeffrey Andrews, and Richard
McElreath. 2024. {``{Bridging theory and data: A computational workflow
for cultural evolution}.''} \emph{Proceedings of the National Academy of
Sciences of the United States of America} 121 (48): 1--11.
\url{https://doi.org/10.1073/pnas.2322887121}.

\bibitem[\citeproctext]{ref-Derex2016}
Derex, Maxime, and Robert Boyd. 2016. {``{Partial connectivity increases
cultural accumulation within groups}.''} \emph{Proceedings of the
National Academy of Sciences of the United States of America} 113 (11):
2982--87. \url{https://doi.org/10.1073/pnas.1518798113}.

\bibitem[\citeproctext]{ref-GalesicEtAl2023}
Galesic, Mirta, Daniel Barkoczi, Andrew M. Berdhal, Dora Biro, Giuseppe
Carbone, Ilaria Giannoccaro, Robert L. Goldstone, et al. 2023.
{``{Beyond collective intelligence : Collective adaptation}.''}
\emph{Journal of the Royal Society Interface} 20 (20220736).
\url{https://doi.org/10.1098/rsif.2022.0736}.

\bibitem[\citeproctext]{ref-Jackson2008}
Jackson, Matthew O. 2008. \emph{{Social and Economic Networks}}.
Princeton: Princeton University Press.
\url{https://press.princeton.edu/books/paperback/9780691148205/social-and-economic-networks}.

\bibitem[\citeproctext]{ref-Kling2024}
Kling, Matthew M., Christopher T. Brittain, Gillian L. Galford, Timothy
M. Waring, Laurent Hébert-Dufresne, Matthew P. Dube, Hossein Sabzian,
Nicholas J. Gotelli, Brian J. McGill, and Meredith T. Niles. 2024.
{``{Innovations through crop switching happen on the diverse margins of
US agriculture}.''} \emph{Proceedings of the National Academy of
Sciences of the United States of America} 121 (42): 1--10.
\url{https://doi.org/10.1073/pnas.2402195121}.

\bibitem[\citeproctext]{ref-Liddell2018}
Liddell, Torrin M., and John K. Kruschke. 2018. {``{Analyzing ordinal
data with metric models: What could possibly go wrong?}''} \emph{Journal
of Experimental Social Psychology} 79: 328--48.
\url{https://doi.org/10.1016/j.jesp.2018.08.009}.

\bibitem[\citeproctext]{ref-Moser2023}
Moser, Cody, and Paul E. Smaldino. 2023. {``{Innovation-facilitating
networks create inequality}.''} \emph{Proceedings of the Royal Society
B: Biological Sciences} 290 (2011).
\url{https://doi.org/10.1098/rspb.2023.2281}.

\bibitem[\citeproctext]{ref-Nowak2006}
Nowak, Martin A. 2006. {``{Evolution of virulence}.''} In
\emph{Evolutionary Dynamics: Exploring the Equations of Life}, 384.
Harvard University Press.
\url{https://doi.org/10.1016/S0891-5520(03)00099-0}.

\bibitem[\citeproctext]{ref-Ostrom2014}
Ostrom, Elinor. 2014. {``{Collective action and the evolution of social
norms}.''} \emph{Journal of Natural Resources Policy Research} 6 (4):
235--52. \url{https://doi.org/10.1080/19390459.2014.935173}.

\bibitem[\citeproctext]{ref-Pisor2024}
Pisor, Anne C., Monique Borgerhoff Mulder, and Kristopher M. Smith.
2024. {``{Long-distance social relationships can both undercut and
promote local natural resource management}.''} \emph{Philosophical
Transactions of the Royal Society B: Biological Sciences} 379 (1893).
\url{https://doi.org/10.1098/rstb.2022.0269}.

\bibitem[\citeproctext]{ref-PisorLansingMagargal2023}
Pisor, Anne, J. Stephen Lansing, and Kate Magargal. 2023. {``{Climate
change adaptation needs a science of culture}.''} \emph{Philosophical
Transactions of the Royal Society B: Biological Sciences} 378 (1889).
\url{https://doi.org/10.1098/rstb.2022.0390}.

\bibitem[\citeproctext]{ref-Richerson2016d}
Richerson, Peter, Ryan Baldini, Adrian V. Bell, Kathryn Demps, Karl
Frost, Vicken Hillis, Sarah Mathew, et al. 2016. {``{Cultural group
selection plays an essential role in explaining human cooperation: A
sketch of the evidence}.''} \emph{Behavioral and Brain Sciences} 39
(2016): e30. \url{https://doi.org/10.1017/S0140525X1400106X}.

\bibitem[\citeproctext]{ref-Ross2024}
Ross, Cody T., Richard McElreath, and Daniel Redhead. 2024.
{``{Modelling animal network data in R using STRAND}.''} \emph{Journal
of Animal Ecology} 93 (3): 254--66.
\url{https://doi.org/10.1111/1365-2656.14021}.

\bibitem[\citeproctext]{ref-Tavoni2011}
Tavoni, Alessandro, Astrid Dannenberg, Giorgos Kallis, and Andreas
Löschel. 2011. {``{Inequality, communication, and the avoidance of
disastrous climate change in a public goods game}.''} \emph{Proceedings
of the National Academy of Sciences of the United States of America} 108
(29): 11825--29. \url{https://doi.org/10.1073/pnas.1102493108}.

\bibitem[\citeproctext]{ref-Turner2023}
Turner, Matthew A, Alyson L Singleton, Mallory J Harris, Ian Harryman,
Cesar Augusto Lopez, Ronan Forde Arthur, Caroline Muraida, and James
Holland Jones. 2023. {``{Minority-group incubators and majority-group
reservoirs support the diffusion of climate change adaptations}.''}

\bibitem[\citeproctext]{ref-Turner2018}
Turner, Matthew A., and Paul E. Smaldino. 2018. {``{Paths to
Polarization: How Extreme Views, Miscommunication, and Random Chance
Drive Opinion Dynamics}.''} \emph{Complexity}.
\url{https://doi.org/10.1155/2018/2740959}.

\bibitem[\citeproctext]{ref-Waring2017}
Waring, Timothy M., Sandra H. Goff, and Paul E. Smaldino. 2017. {``{The
coevolution of economic institutions and sustainable consumption via
cultural group selection}.''} \emph{Ecological Economics} 131: 524--32.
\url{https://doi.org/10.1016/j.ecolecon.2016.09.022}.

\bibitem[\citeproctext]{refWaring2015}
Waring, Timothy M., Michelle Ann Kline, Jeremy S. Brooks, Sandra H.
Goff, John Gowdy, Marco A. Janssen, Paul E. Smaldino, and Jennifer
Jacquet. 2015. {``{A multilevel evolutionary framework for
sustainability analysis}.''} \emph{Ecology and Society} 20 (2).
\url{https://doi.org/10.5751/ES-07634-200234}.

\bibitem[\citeproctext]{ref-Waring2023}
Waring, Timothy M., Meredith T. Niles, Matthew M. Kling, Stephanie N.
Miller, Laurent Hébert-Dufresne, Hossein Sabzian, Nicholas Gotelli, and
Brian J. McGill. 2023. {``{Operationalizing cultural adaptation to
climate change: Contemporary examples from United States
agriculture}.''} \emph{Philosophical Transactions of the Royal Society
B: Biological Sciences} 378 (1889).
\url{https://doi.org/10.1098/rstb.2022.0397}.

\end{CSLReferences}

\end{document}
